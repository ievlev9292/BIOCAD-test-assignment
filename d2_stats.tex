\documentclass[12pt, letterpaper, twoside]{article}
\usepackage[T2A]{fontenc}
\usepackage[utf8]{inputenc}
\usepackage[russian]{babel}
\usepackage{hyphenat}
 

 
\begin{document}
 
\paragraph{Задача 2}
\begin{quotation} 
	Есть 3 эксперта, которые прогнозируют победу России в матче с Испанией на ЧМ 2018! Каждый из экспертов выставляет верный прогноз (прогноз бинарен: победа России или нет) с вероятностью 90 процентов. Какова будет вероятность верного прогноза в случае, если прогноз будет осуществляться сразу тремя экспертами, а финальное решение будет осуществляться голосованием, а именно правилом большинства.	
\end{quotation}
 
\paragraph{Решение}

Решение каждого эксперта есть случайная величина Бернулли Ber($p$), принимающая значения: 1 -- верный прогноз, 0 -- ошибочный прогноз, где $p$ -- вероятность верного прогноза, $p=0.9$.
Верный прогноз при голосовании трех экспертов возможен при следующих комбинациях случайных величин: 111, 110, 101, 011. Вычислим вероятность такого события $P=0.9^3+3 \cdot 0.9^2 \cdot 0.1=0.972$.
Таким образом, вероятность верного прогноза в случае, если прогноз будет осуществляться сразу тремя экспертами, а финальное решение будет осуществляться голосованием, а именно правилом большинства равняется $0.972$.
 
\end{document}